\nonstopmode{}
\documentclass[a4paper]{book}
\usepackage[times,inconsolata,hyper]{Rd}
\usepackage{makeidx}
\makeatletter\@ifl@t@r\fmtversion{2018/04/01}{}{\usepackage[utf8]{inputenc}}\makeatother
% \usepackage{graphicx} % @USE GRAPHICX@
\makeindex{}
\begin{document}
\chapter*{}
\begin{center}
{\textbf{\huge Package `MetabolomicsPipeline'}}
\par\bigskip{\large \today}
\end{center}
\ifthenelse{\boolean{Rd@use@hyper}}{\hypersetup{pdftitle = {MetabolomicsPipeline: Metabolomics Pipeline Tools}}}{}
\ifthenelse{\boolean{Rd@use@hyper}}{\hypersetup{pdfauthor = {Joel Parker; Bonnie Lafleur}}}{}
\begin{description}
\raggedright{}
\item[Title]\AsIs{Metabolomics Pipeline Tools}
\item[Version]\AsIs{0.99.0}
\item[Description]\AsIs{This package provides analysis tools for analyzing metabolomics data from Metabolon. The tools
in this package compliment the analysis from Metabolon. We provide functionality for hypothesis testing at
the subpathway level, pairwise comparisons of metabolites, and tools for exploratory analysis. }
\item[biocViews]\AsIs{Metabolomics, Software}
\item[License]\AsIs{MIT + file LICENSE}
\item[Encoding]\AsIs{UTF-8}
\item[Roxygen]\AsIs{list(markdown = TRUE)}
\item[RoxygenNote]\AsIs{7.3.1}
\item[Suggests]\AsIs{ggplotify, magick, openxlsx, rmarkdown, table1, testthat (>=
3.0.0)}
\item[Config/testthat/edition]\AsIs{3}
\item[Imports]\AsIs{dplyr (>= 1.1.3), emmeans (>= 1.8.8), factoextra (>= 1.0.7),
FactoMineR (>= 2.8), ggplot2 (>= 3.4.3), grDevices (>= 3.6.0),
kableExtra (>= 1.3.4), knitr (>= 1.44), magrittr, methods (>=
4.3.1), pheatmap (>= 1.0.12), plotly (>= 4.10.2), RColorBrewer
(>= 1.1.3), readxl, reshape2 (>= 1.4.4), stats, stringr (>=
1.5.0), SummarizedExperiment, tibble, tidyr (>= 1.3.0)}
\item[Depends]\AsIs{R (>= 4.4.0)}
\item[Maintainer]\AsIs{Joel Parker }\email{joelparker@arizona.edu}\AsIs{}
\item[Author]\AsIs{Bonnie Lafleur }\email{blafleur@arizona.edu}\AsIs{}
\item[VignetteBuilder]\AsIs{knitr}
\item[URL]\AsIs{}\url{https://github.com/JoelParkerUofA/MetabolomicsPipeline}\AsIs{}
\item[BugReports]\AsIs{}\url{https://github.com/JoelParkerUofA/MetabolomicsPipeline/issues}\AsIs{}
\item[NeedsCompilation]\AsIs{no}
\end{description}
\Rdcontents{Contents}
\HeaderA{MetabolomicsPipeline-package}{MetabolomicsPipeline: Metabolomics Pipeline Tools}{MetabolomicsPipeline.Rdash.package}
\aliasA{MetabolomicsPipeline}{MetabolomicsPipeline-package}{MetabolomicsPipeline}
\keyword{internal}{MetabolomicsPipeline-package}
%
\begin{Description}
This package provides analysis tools for analyzing metabolomics data from Metabolon. The tools in this package compliment the analysis from Metabolon. We provide functionality for hypothesis testing at the subpathway level, pairwise comparisons of metabolites, and tools for exploratory analysis.
\end{Description}
%
\begin{Author}
\strong{Maintainer}: Joel Parker \email{joelparker@arizona.edu} (\Rhref{https://orcid.org/0000-0002-3411-3818}{ORCID})

Authors:
\begin{itemize}

\item{} Bonnie Lafleur \email{blafleur@arizona.edu}

\end{itemize}


\end{Author}
%
\begin{SeeAlso}
Useful links:
\begin{itemize}

\item{} \url{https://github.com/JoelParkerUofA/MetabolomicsPipeline}
\item{} Report bugs at \url{https://github.com/JoelParkerUofA/MetabolomicsPipeline/issues}

\end{itemize}


\end{SeeAlso}
\HeaderA{all\_sig\_subpath}{Table of Significant Subpathways}{all.Rul.sig.Rul.subpath}
%
\begin{Description}
Create a table of all significant subpathways
\end{Description}
%
\begin{Usage}
\begin{verbatim}
all_sig_subpath(path_results)
\end{verbatim}
\end{Usage}
%
\begin{Arguments}
\begin{ldescription}
\item[\code{path\_results}] Results data frame generated by
\code{\LinkA{subpathway\_analysis}{subpathway.Rul.analysis}}
\end{ldescription}
\end{Arguments}
%
\begin{Value}
A table of all significant subpathways. Including the significant
model type
and model type p-value.
\end{Value}
%
\begin{Examples}
\begin{ExampleCode}
# Load data
data("demoDataSmall", package = "MetabolomicsPipeline")
dat <- demoDataSmall

# Runsubpathay analysis
sub_analysis <- subpathway_analysis(dat,
    treat_var = "GROUP_NAME",
    block_var = "TIME1",
    strat_var = NULL,
    Assay = "normalized"
)

##############################################################################
### Results Plots ############################################################
##############################################################################

# significant subpathways by model type
subpath_by_model(sub_analysis)

# Percentage of signficant subpathways within superpathways
subpath_within_superpath(sub_analysis)

met_within_sub(sub_analysis, subpathway = "Aminosugar Metabolism")

# All signifiicant subpathways
all_sig_subpath(sub_analysis)

\end{ExampleCode}
\end{Examples}
\HeaderA{create\_heatmap\_Data}{Create metadata and matricies for metabolite heatmaps}{create.Rul.heatmap.Rul.Data}
%
\begin{Description}
This function creates the required matrices for the metabolite heatmaps.
\end{Description}
%
\begin{Usage}
\begin{verbatim}
create_heatmap_Data(data, heatmap_variables, Assay = "normalized", ...)
\end{verbatim}
\end{Usage}
%
\begin{Arguments}
\begin{ldescription}
\item[\code{data}] A SummarizedExperiment containing Metabolon data.

\item[\code{heatmap\_variables}] A vector of variable names that are NOT metabolites.

\item[\code{Assay}] Name of assay data to be used for heatmaps.
Default="normalized".

\item[\code{...}] Additional arguments that can be passed into the arrange function.
This parameter will order the columns of the heatmap data.
\end{ldescription}
\end{Arguments}
%
\begin{Value}
A list of matrices including the heatmap variable
(meta data for heatmap)
and the values for the heatmap.
\end{Value}
\HeaderA{demoDat}{Demo data for the MetabolomicsPipeline,}{demoDat}
\keyword{data}{demoDat}
%
\begin{Description}
Demo data consisting of 86 samples (42 males, 44 females), three treatment
groups, and the samples were taken
\end{Description}
%
\begin{Usage}
\begin{verbatim}
demoDat
\end{verbatim}
\end{Usage}
%
\begin{Format}
SummarizedExperiment object
\end{Format}
%
\begin{Value}
A SummarizedExperiment object with 86 samples
\end{Value}
\HeaderA{demoDataSmall}{Subset of Demo data for the MetabolomicsPipeline ,}{demoDataSmall}
\keyword{data}{demoDataSmall}
%
\begin{Description}
Demo data consisting of 86 samples (42 males, 44 females), three treatment
groups, and the samples were taken at three different time points.
We focus on a subset of 10 Subpathways.
\end{Description}
%
\begin{Usage}
\begin{verbatim}
demoDataSmall
\end{verbatim}
\end{Usage}
%
\begin{Format}
Rd
\end{Format}
%
\begin{Value}
A subset of the metabolites in the DemoData.
\end{Value}
\HeaderA{loadMetabolon}{Load Metabolomic Data as SummarizedExperiment}{loadMetabolon}
%
\begin{Description}
Automatically load metabolomic data from Metabolon
\end{Description}
%
\begin{Usage}
\begin{verbatim}
loadMetabolon(
  path,
  raw_sheet = "Peak Area Data",
  chemical_sheet = "Chemical Annotation",
  sample_meta = "Sample Meta Data",
  normalized_peak = "Log Transformed Data",
  sample_names = "PARENT_SAMPLE_NAME",
  chemicalID = "CHEM_ID"
)
\end{verbatim}
\end{Usage}
%
\begin{Arguments}
\begin{ldescription}
\item[\code{path}] Path to Metabolon .xlsx file containg peak data, chemical
annotations, sample meta data, and (optionally) the normalized peak counts

\item[\code{raw\_sheet}] Sheet name for the raw peak data.

\item[\code{chemical\_sheet}] Sheet name for chemical annotation.

\item[\code{sample\_meta}] Sheet name for sample meta data.

\item[\code{normalized\_peak}] Sheet name for the normalized peak data.
If you are not adding the normalized data from the excel file then set
normalized\_peak=NA.

\item[\code{sample\_names}] Column name in the meta data containing the sample names.
This must correspond to the row names of the raw peak data in the excel
file.

\item[\code{chemicalID}] Column name in the meta data containing the sample names.
This must correspond to the column names of the raw peak data.
\end{ldescription}
\end{Arguments}
%
\begin{Details}
The Metabolon experiment data are stored in a SummarizedExperiment.
\end{Details}
%
\begin{Value}
A SummarizedExperiment containing Metabolon expirement data.
\end{Value}
%
\begin{SeeAlso}
\LinkA{SummarizedExperiment::SummarizedExperiment}{SummarizedExperiment::SummarizedExperiment}
\end{SeeAlso}
\HeaderA{log\_transformation}{Log transformation of metabolite data}{log.Rul.transformation}
%
\begin{Description}
This function log transforms each metabolite in the Metabolon data.
\end{Description}
%
\begin{Usage}
\begin{verbatim}
log_transformation(peak_data)
\end{verbatim}
\end{Usage}
%
\begin{Arguments}
\begin{ldescription}
\item[\code{peak\_data}] A matrix of peak data with metabolites in the columns
\end{ldescription}
\end{Arguments}
%
\begin{Value}
log transformed peak data
\end{Value}
%
\begin{Examples}
\begin{ExampleCode}
data("demoDataSmall", package = "MetabolomicsPipeline")
peak <- SummarizedExperiment::assay(demoDataSmall, "peak")

# Median standardization
peak_med <- median_standardization(peak_data = peak)

# Min value imputation
peakImpute <- min_val_impute(peak_data = peak_med)

# log transformation
peak_log <- log_transformation(peak_data = peakImpute)


\end{ExampleCode}
\end{Examples}
\HeaderA{median\_standardization}{Median standardization for metabolite data}{median.Rul.standardization}
%
\begin{Description}
This function standardizes the metabolites by the median of the metabolite.
\end{Description}
%
\begin{Usage}
\begin{verbatim}
median_standardization(peak_data)
\end{verbatim}
\end{Usage}
%
\begin{Arguments}
\begin{ldescription}
\item[\code{peak\_data}] Peak data with metabolites in the columns. The data also
must include the "PARENT\_SAMPLE\_NAME".
\end{ldescription}
\end{Arguments}
%
\begin{Value}
Median standardized peak data.
\end{Value}
%
\begin{Examples}
\begin{ExampleCode}
data("demoDataSmall", package = "MetabolomicsPipeline")
peak <- SummarizedExperiment::assay(demoDataSmall, "peak")

# Median standardization
peak_med <- median_standardization(peak_data = peak)

# Min value imputation
peakImpute <- min_val_impute(peak_data = peak_med)

# log transformation
peak_log <- log_transformation(peak_data = peakImpute)



\end{ExampleCode}
\end{Examples}
\HeaderA{metabolite\_heatmap}{Create metabolite heatmap}{metabolite.Rul.heatmap}
%
\begin{Description}
Create heatmaps which are arranged by the experimental conditions.
\end{Description}
%
\begin{Usage}
\begin{verbatim}
metabolite_heatmap(
  data,
  top_mets = 50,
  group_vars,
  strat_var = NULL,
  caption = NULL,
  Assay = "normalized",
  ...
)
\end{verbatim}
\end{Usage}
%
\begin{Arguments}
\begin{ldescription}
\item[\code{data}] A SummarizedExperiment containing the Metabolon experiment data.

\item[\code{top\_mets}] Number of metabolites to include in the heatmap. Metabolites
are chosen based on the highest variability.

\item[\code{group\_vars}] Vector of variables to annotate heatmap with. Columns will
be grouped by these variables.

\item[\code{strat\_var}] Variable to stratify the heatmap by.

\item[\code{caption}] A title for the heatmap. If strat\_var is used, the title will
automatically include the stratum with the tile.

\item[\code{Assay}] Which assay data to use for the heatmap (default="normalized").

\item[\code{...}] Additional arguments can be passed into the arrange function.
This parameter will order the columns of the heatmap.
\end{ldescription}
\end{Arguments}
%
\begin{Value}
A gtable class with all of the information to build the heatmap.
To view the heatmap use ggplotify::as.ggplot().
\end{Value}
%
\begin{Examples}
\begin{ExampleCode}
# load data
data("demoDat", package = "MetabolomicsPipeline")
dat <- demoDat

# Heatmap with one group
treat_heatmap <- metabolite_heatmap(dat,
    top_mets = 50,
    group_vars = "GROUP_NAME",
    strat_var = NULL,
    caption = "Heatmap Arranged By Group",
    Assay = "normalized",
    GROUP_NAME
)

\end{ExampleCode}
\end{Examples}
\HeaderA{metabolite\_pairwise}{Metabolite Pairwise Comparisons.}{metabolite.Rul.pairwise}
%
\begin{Description}
Computes the pairwise comparison estimates and p-values for each metabolite.
\end{Description}
%
\begin{Usage}
\begin{verbatim}
metabolite_pairwise(
  data,
  form,
  Assay = "normalized",
  strat_var = NULL,
  mets = NULL
)
\end{verbatim}
\end{Usage}
%
\begin{Arguments}
\begin{ldescription}
\item[\code{data}] SummarizedExperiment with Metabolon experiment data.

\item[\code{form}] This is a character string the resembles the right hand side of
a simple linear regression model in R. For example form = "Group1 + Group2".

\item[\code{Assay}] Name of the assay to be used for the pairwise analysis
(default='normalized')

\item[\code{strat\_var}] A variable in the analysis data to stratify the model by.
If this is specified, a list of results will be returned.

\item[\code{mets}] Chemical ID for the metabolites of interest. If NULL then the
pairwise analysis is completed for all metabololites.
\end{ldescription}
\end{Arguments}
%
\begin{Details}
This function will analyze each metabolite individually. For each
metabolite, the metabolite\_pairwise function
will first test whether the model explained a significant proportion
of the variance in the metabolite using an F-test. Since we will be looking
at multiple comparisons for the metabolite, it is good practice to first look
at the overall p-value from the F-test before looking at the pairwise
comparisons.
The metabolite\_pairwise function then looks at all pairwise comparisons
utilizing the
\Rhref{https://cran.r-project.org/web/packages/emmeans/index.html}{emmeans}
package. The metabolite\_pairwise function returns a data frame with the
metabolite overall p-value, log fold change for each group, and the p-value
for each comparison.
\end{Details}
%
\begin{Value}
The overall F-test p-value, and the estimate and pvalue for each
pairwise comparison.
\end{Value}
%
\begin{Examples}
\begin{ExampleCode}
# Load data
data("demoDat", package = "MetabolomicsPipeline")
dat <- demoDat

# Run pairwise analysis
strat_pairwise <- metabolite_pairwise(dat,
    form = "GROUP_NAME*TIME1",
    strat_var = "Gender"
)

\end{ExampleCode}
\end{Examples}
\HeaderA{metabolite\_pca}{Metabolite PCA}{metabolite.Rul.pca}
%
\begin{Description}
Computes and plots the first two components of the PCA from the metabolite
data.
\end{Description}
%
\begin{Usage}
\begin{verbatim}
metabolite_pca(data, Assay = "normalized", meta_var)
\end{verbatim}
\end{Usage}
%
\begin{Arguments}
\begin{ldescription}
\item[\code{data}] SummarizedExperiment with Metabolon experiment data.

\item[\code{Assay}] Name of the assay to be used for the pairwise analysis
(default='normalized')

\item[\code{meta\_var}] A metadata variable to color code the PCA plot by.
\end{ldescription}
\end{Arguments}
%
\begin{Value}
A PCA plot of the first two principal components, colored by the
metadata variable.
\end{Value}
%
\begin{Examples}
\begin{ExampleCode}

# load data
data("demoDat", package = "MetabolomicsPipeline")
dat <- demoDat

# Define PCA label from metadata
meta_var <- "Gender"

# Run PCA
pca <- metabolite_pca(dat,
    meta_var = meta_var
)


# Show PCA
pca

\end{ExampleCode}
\end{Examples}
\HeaderA{met\_est\_heatmap}{Metabolite Pairwise Estimate Interactive Heatmap.}{met.Rul.est.Rul.heatmap}
%
\begin{Description}
Produce an interactive heatmap of the estimates produced in
\code{\LinkA{metabolite\_pairwise}{metabolite.Rul.pairwise}}.
\end{Description}
%
\begin{Usage}
\begin{verbatim}
met_est_heatmap(
  results_data,
  data,
  interactive = FALSE,
  CHEM_ID = "CHEM_ID",
  SUB_PATHWAY = "SUB_PATHWAY",
  CHEMICAL_NAME = "CHEMICAL_NAME",
  ...
)
\end{verbatim}
\end{Usage}
%
\begin{Arguments}
\begin{ldescription}
\item[\code{results\_data}] Results data frame of the pairwise comparisons produced
by \code{\LinkA{metabolite\_pairwise}{metabolite.Rul.pairwise}}.

\item[\code{data}] A SummarizedExperiment containing the Metabolon experiment data.

\item[\code{interactive}] boolean (T/F) for whether or not the plot should be
interactive. Use interactive=T to produce an interactive plot using
plotly. Use interactive=F to produce a static heatmap using pheatmap.

\item[\code{CHEM\_ID}] Column name in the chemical annotation worksheet that contains
the chemical ID.

\item[\code{SUB\_PATHWAY}] Column name in the chemical annotation worksheet which
contains the subpathway information.

\item[\code{CHEMICAL\_NAME}] Column name in the chemical annotation worksheet which
contains the chemical name.

\item[\code{...}] Additional arguments that can be passed to pheatmap.
\end{ldescription}
\end{Arguments}
%
\begin{Details}
This function will produce a heatmap of the log fold changes for the
metabolites with a significant overall p-value (which tested if the treatment
group means were equal under the null hypothesis). The heatmap colors will
only show if the log fold-change is greater than log(2) or less than
log(.5). Therefore, this heatmap will only focus on comparisons with a
fold change of two or greater.
\end{Details}
%
\begin{Value}
An interactive heatmap of pairwise estimates.
\end{Value}
\HeaderA{met\_p\_heatmap}{Metabolite Pairwise P-Value Interactive Heatmap.}{met.Rul.p.Rul.heatmap}
%
\begin{Description}
Produce an interactive heatmap of the p-values produced in
\code{\LinkA{metabolite\_pairwise}{metabolite.Rul.pairwise}}.
\end{Description}
%
\begin{Usage}
\begin{verbatim}
met_p_heatmap(
  results_data,
  data,
  interactive = FALSE,
  CHEM_ID = "CHEM_ID",
  SUB_PATHWAY = "SUB_PATHWAY",
  CHEMICAL_NAME = "CHEMICAL_NAME",
  ...
)
\end{verbatim}
\end{Usage}
%
\begin{Arguments}
\begin{ldescription}
\item[\code{results\_data}] Results data frame of the pairwise comparisons produced
by \code{\LinkA{metabolite\_pairwise}{metabolite.Rul.pairwise}}.

\item[\code{data}] A SummarizedExperiment containing Metabolon experiment data.

\item[\code{interactive}] boolean (T/F) for whether or not the plot should be
interactive. Use interactive=T to produce an interactive plot using
plotly. Use interactive=F to produce a static heatmap using pheatmap.

\item[\code{CHEM\_ID}] Column name in the chemical annotation worksheet that contains
the chemical ID.

\item[\code{SUB\_PATHWAY}] Column name in the chemical annotation worksheet which
contains the subpathway information.

\item[\code{CHEMICAL\_NAME}] Column name in the chemical annotation worksheet which
contains the chemical name.

\item[\code{...}] Additional arguments that can be passed to pheatmap.
\end{ldescription}
\end{Arguments}
%
\begin{Details}
For the metabolites which had a significant overall p-value (which tested if
the treatment group means were equal under the null hypothesis), we will
produce a heatmap of the p-values.
\end{Details}
%
\begin{Value}
An interactive heatmap of pairwise p-values.
\end{Value}
\HeaderA{met\_within\_sub}{Metabolites within Subpathway Table}{met.Rul.within.Rul.sub}
%
\begin{Description}
Return the model results for each metabolite within a subpathway.
\end{Description}
%
\begin{Usage}
\begin{verbatim}
met_within_sub(
  subpath_results,
  subpathway,
  mod = c("interaction", "parallel", "single")
)
\end{verbatim}
\end{Usage}
%
\begin{Arguments}
\begin{ldescription}
\item[\code{subpath\_results}] Results data frame generated by
\code{\LinkA{subpathway\_analysis}{subpathway.Rul.analysis}}

\item[\code{subpathway}] Character string of the subpathway of interest.
This is case sensitive and must be listed in the subpath\_results.

\item[\code{mod}] Model of interest. This can be a single model or a vector of
model types that can take on the values "interaction", "parallel",
or "single".
\end{ldescription}
\end{Arguments}
%
\begin{Value}
A table with the results from the model types specified and for each
metabolite within the superpathway specified.
\end{Value}
%
\begin{Examples}
\begin{ExampleCode}
data("demoDataSmall", package = "MetabolomicsPipeline")
dat <- demoDataSmall

# Runsubpathay analysis
sub_analysis <- subpathway_analysis(dat,
    treat_var = "GROUP_NAME",
    block_var = "TIME1",
    strat_var = NULL,
    Assay = "normalized"
)

#############################################################################
### Results Plots ###########################################################
#############################################################################

# significant subpathways by model type
subpath_by_model(sub_analysis)

# Percentage of signficant subpathways within superpathways
subpath_within_superpath(sub_analysis)

met_within_sub(sub_analysis, subpathway = "Aminosugar Metabolism")

\end{ExampleCode}
\end{Examples}
\HeaderA{min\_val\_impute}{Minimum Value Imputation}{min.Rul.val.Rul.impute}
%
\begin{Description}
Imputes the minimum value for each metabolite
\end{Description}
%
\begin{Usage}
\begin{verbatim}
min_val_impute(peak_data)
\end{verbatim}
\end{Usage}
%
\begin{Arguments}
\begin{ldescription}
\item[\code{peak\_data}] Peak data matrix with metabolites in the columns.
\end{ldescription}
\end{Arguments}
%
\begin{Value}
Metabolite imputed peak data.
\end{Value}
%
\begin{Examples}
\begin{ExampleCode}
data("demoDataSmall", package = "MetabolomicsPipeline")
peak <- SummarizedExperiment::assay(demoDataSmall, "peak")

# Median standardization
peak_med <- median_standardization(peak_data = peak)

# Min value imputation
peakImpute <- min_val_impute(peak_data = peak_med)

# log transformation
peak_log <- log_transformation(peak_data = peakImpute)

\end{ExampleCode}
\end{Examples}
\HeaderA{pairwise}{Pairwise function}{pairwise}
%
\begin{Description}
This is the main function for metabolite\_pairwise
\end{Description}
%
\begin{Usage}
\begin{verbatim}
pairwise(out, form, data)
\end{verbatim}
\end{Usage}
%
\begin{Arguments}
\begin{ldescription}
\item[\code{out}] Outcome used as reponse

\item[\code{form}] form of the model

\item[\code{data}] data used for modeling
\end{ldescription}
\end{Arguments}
%
\begin{Value}
Pairwise comparisons for a single metabolite.
\end{Value}
\HeaderA{subpathway\_analysis}{Subpathway Analysis}{subpathway.Rul.analysis}
%
\begin{Description}
Subpathway analysis for metabolite data.
\end{Description}
%
\begin{Usage}
\begin{verbatim}
subpathway_analysis(
  data,
  treat_var,
  block_var = NULL,
  strat_var = NULL,
  Assay = "normalized",
  subPathwayName = "SUB_PATHWAY",
  chemName = "CHEMICAL_NAME",
  superPathwayName = "SUPER_PATHWAY"
)
\end{verbatim}
\end{Usage}
%
\begin{Arguments}
\begin{ldescription}
\item[\code{data}] SummarizedExperiment with Metabolon experiment data.

\item[\code{treat\_var}] This is the name of the variable in the analysis data that
is the main variable of interest.

\item[\code{block\_var}] This is the name of the blocking variable in the dataset.
If the the experimental design does not include a blocking variable, then the
value of block\_var=NULL.

\item[\code{strat\_var}] Variable to stratify the subpathway analysis by. This is set
to NULL by default and will not stratify the analysis unless specified.

\item[\code{Assay}] Name of the assay to be used for the pairwise analysis
(default='normalized')

\item[\code{subPathwayName}] Column name for subpathway variable as defined in the
chemical annotation worksheet.

\item[\code{chemName}] Column name for chemical name variable as defined in the
chemical annotation worksheet.

\item[\code{superPathwayName}] Column name for super-pathway variable as defined in
the chemical annotation worksheet.
\end{ldescription}
\end{Arguments}
%
\begin{Details}
For each metabolite, we test three models using using ANOVA.
\begin{enumerate}


\item{} Interaction:  \eqn{log Peak = Treatment + block + Treatment*block}{}

\item{} Parallel: \eqn{log Peak = Treatment + block}{}

\item{} Single: \eqn{log Peak =  Treatment}{}


\end{enumerate}


For the interaction model, we are focusing only on the interaction term
"Treatment*block" to test if there is a significant interaction between our
treatment and the block variable. The parallel model tests if the block
variable explains a significant amount of the metabolite variance, and the
treatment model tests if the treatment explains a significant proportion of
the variance for each metabolite. Then, we use the Combined Fisher
probability to test each model at the subpathway level.

\deqn{\tilde{X} = -2\sum_{i=1}^k ln(p_i)}{}

where \eqn{k}{} is the number of metabolites in the subpathway. We can
get a p-value from \eqn{P(X \geq\tilde{X})}{}, knowing that
\eqn{\tilde{X}\sim \chi^2_{2k}}{}. You will notice that smaller p-values will
lead to a larger \eqn{\tilde{X}}{}.
\end{Details}
%
\begin{Value}
A data frame with "CHEM\_ID","sub\_pathway","chem\_name",
"interaction\_pval","interaction\_fisher","parallel\_pval","parallel\_fisher",
"single\_pval","single\_fisher",and "model" for each metabolite.
\end{Value}
%
\begin{SeeAlso}
\Rhref{https://www.sciencedirect.com/science/article/pii/S0167947303002950}{Loughin, Thomas M. "A systematic comparison of methods for combining p-values from independent tests." Computational statistics \& data analysis 47.3 (2004): 467-485.}
\end{SeeAlso}
%
\begin{Examples}
\begin{ExampleCode}
# Load data
data("demoDataSmall", package = "MetabolomicsPipeline")
dat <- demoDataSmall

# Runsubpathay analysis
sub_analysis <- subpathway_analysis(dat,
    treat_var = "GROUP_NAME",
    block_var = "TIME1",
    strat_var = NULL,
    Assay = "normalized"
)

##############################################################################
### Results Plots ###########################################################
#############################################################################

# significant subpathways by model type
subpath_by_model(sub_analysis)

# Percentage of signficant subpathways within superpathways
subpath_within_superpath(sub_analysis)

met_within_sub(sub_analysis, subpathway = "Aminosugar Metabolism")

# All signifiicant subpathways
all_sig_subpath(sub_analysis)

\end{ExampleCode}
\end{Examples}
\HeaderA{subpathway\_boxplots}{Subpathway Boxplots}{subpathway.Rul.boxplots}
%
\begin{Description}
Creates boxplots for each metabolite within a specified subpathway.
\end{Description}
%
\begin{Usage}
\begin{verbatim}
subpathway_boxplots(
  data,
  subpathway,
  block_var,
  treat_var,
  Assay = "normalized",
  CHEMICAL_NAME = "CHEMICAL_NAME",
  CHEM_ID = "CHEM_ID",
  SUB_PATHWAY = "SUB_PATHWAY",
  ...
)
\end{verbatim}
\end{Usage}
%
\begin{Arguments}
\begin{ldescription}
\item[\code{data}] SummarizedExperiment with Metabolon experiment data.

\item[\code{subpathway}] Character value of the subpathway of interest. This is case
sensitive and must be in the chemical annotation file.

\item[\code{block\_var}] This the the name of the variable in the meta data that is
used for the X axis of the box plots. We recommend using the "block\_var"
from the subpathway analysis.

\item[\code{treat\_var}] This is a grouping variable. As a recommendation the
treatment groups should be used in the treat\_var argument as this will
provide a different color for each of the treatments making it easier to
identify.

\item[\code{Assay}] Name of the assay to be used for the pairwise analysis
(default='normalized')

\item[\code{CHEMICAL\_NAME}] Column name in the chemical annotation worksheet which
contains the chemical name.

\item[\code{CHEM\_ID}] Column name in the chemical annotation worksheet that contains
the chemical ID.

\item[\code{SUB\_PATHWAY}] Column name in chemical annotation file which contains
the SUB\_PATHWAY information

\item[\code{...}] Additional arguments to filter the analysis data by.
\end{ldescription}
\end{Arguments}
%
\begin{Details}
.
\end{Details}
%
\begin{Value}
Boxplots stratified by metabolites.
\end{Value}
%
\begin{Examples}
\begin{ExampleCode}
# load data
data("demoDat", package = "MetabolomicsPipeline")
dat <- demoDat

#############################################################################
### BoxPlots ###############################################################
############################################################################

subpathway_boxplots(dat,
                 subpathway = "Lactoyl Amino Acid", block_var = TIME1,
                 treat_var = GROUP_NAME, Assay = "normalized",
                 CHEMICAL_NAME = "CHEMICAL_NAME",
                 CHEM_ID="CHEM_ID",
                 SUB_PATHWAY="SUB_PATHWAY",Gender == "Female")


############################################################################
## Line plots ##############################################################
############################################################################

# Set up data
dat$TIME1 <- as.numeric(factor(dat$TIME1,
    levels = c("PreSymp", "Onset", "End")
))
# Create line plots
subpathway_lineplots(dat,
                    subpathway = "Lactoyl Amino Acid",
                    block_var = TIME1, treat_var = GROUP_NAME,
                    Assay = "normalized",
                    CHEMICAL_NAME = "CHEMICAL_NAME",
                    CHEM_ID="CHEM_ID",
                    SUB_PATHWAY="SUB_PATHWAY",Gender == "Female")



\end{ExampleCode}
\end{Examples}
\HeaderA{subpathway\_lineplots}{Subpathway Lineplots}{subpathway.Rul.lineplots}
%
\begin{Description}
Create line plots for each metabolite within a subpathway.
\end{Description}
%
\begin{Usage}
\begin{verbatim}
subpathway_lineplots(
  data,
  subpathway,
  block_var,
  treat_var,
  Assay = "normalized",
  CHEMICAL_NAME = "CHEMICAL_NAME",
  CHEM_ID = "CHEM_ID",
  SUB_PATHWAY = "SUB_PATHWAY",
  ...
)
\end{verbatim}
\end{Usage}
%
\begin{Arguments}
\begin{ldescription}
\item[\code{data}] SummarizedExperiment with Metabolon experiment data.

\item[\code{subpathway}] Character value of the subpathway of interest. This is case
sensitive and must be in the chemical annotation file.

\item[\code{block\_var}] This the the name of the variable in the meta data that is
used for the X axis of the line plots. We recommend using the "block\_var"
variable from the subpathway analyis.

\item[\code{treat\_var}] This is a grouping variable. As a recommendation the
treatment groups should be used in the groupBy argument as this will provide
a different color for each of the treatments making it easier to identify.

\item[\code{Assay}] Name of the assay to be used for the pairwise analysis
(default='normalized')

\item[\code{CHEMICAL\_NAME}] Column name in the chemical annotation worksheet which
contains the chemical name.

\item[\code{CHEM\_ID}] Column name in the chemical annotation worksheet that contains
the chemical ID.

\item[\code{SUB\_PATHWAY}] Column name in the chemical annotation worksheet which
contains the subpathway information.

\item[\code{...}] Additional arguments to filter the analysis data by.
\end{ldescription}
\end{Arguments}
%
\begin{Value}
Line plots stratified by metabolite.
\end{Value}
%
\begin{Examples}
\begin{ExampleCode}
data("demoDat", package = "MetabolomicsPipeline")
dat <- demoDat

#############################################################################
### BoxPlots ###############################################################
############################################################################

subpathway_boxplots(dat,
                 subpathway = "Lactoyl Amino Acid", block_var = TIME1,
                 treat_var = GROUP_NAME, Assay = "normalized",
                 CHEMICAL_NAME = "CHEMICAL_NAME",
                 CHEM_ID="CHEM_ID",
                 SUB_PATHWAY="SUB_PATHWAY",Gender == "Female")


############################################################################
## Line plots ##############################################################
############################################################################

# Set up data
dat$TIME1 <- as.numeric(factor(dat$TIME1,
    levels = c("PreSymp", "Onset", "End")
))

# Create line plots
subpathway_lineplots(dat,
                    subpathway = "Lactoyl Amino Acid",
                    block_var = TIME1, treat_var = GROUP_NAME,
                    Assay = "normalized",
                    CHEMICAL_NAME = "CHEMICAL_NAME",
                    CHEM_ID="CHEM_ID",
                    SUB_PATHWAY="SUB_PATHWAY",Gender == "Female")




\end{ExampleCode}
\end{Examples}
\HeaderA{subpath\_by\_model}{Subpathway model type table}{subpath.Rul.by.Rul.model}
%
\begin{Description}
Create a table with the number of significant subpathways for each model
type.
\end{Description}
%
\begin{Usage}
\begin{verbatim}
subpath_by_model(subpath_results)
\end{verbatim}
\end{Usage}
%
\begin{Arguments}
\begin{ldescription}
\item[\code{subpath\_results}] Results data frame generated by
\code{\LinkA{subpathway\_analysis}{subpathway.Rul.analysis}}
\end{ldescription}
\end{Arguments}
%
\begin{Details}
Each subpathway will only have one model type. We first test the interaction,
and then the parallel and single models are tested last. Suppose a
subpathway has a significant interaction model type. In that case, the table
will count it as an interaction and not as a parallel or single.
\end{Details}
%
\begin{Value}
A table of the number of significant subpathways by model type.
\end{Value}
%
\begin{Examples}
\begin{ExampleCode}
# Load data
data("demoDataSmall", package = "MetabolomicsPipeline")
dat <- demoDataSmall

# Runsubpathay analysis
sub_analysis <- subpathway_analysis(dat,
    treat_var = "GROUP_NAME",
    block_var = "TIME1",
    strat_var = NULL,
    Assay = "normalized"
)

#############################################################################
### Results Plots ###########################################################
#############################################################################

# significant subpathways by model type
subpath_by_model(sub_analysis)

# Percentage of signficant subpathways within superpathways
subpath_within_superpath(sub_analysis)

met_within_sub(sub_analysis, subpathway = "Aminosugar Metabolism")

\end{ExampleCode}
\end{Examples}
\HeaderA{subpath\_within\_superpath}{Proportion of the Significant Subpathways Within Superpathways}{subpath.Rul.within.Rul.superpath}
%
\begin{Description}
Create a table that gives the percentage of significant subpathways within
each superpathway.
\end{Description}
%
\begin{Usage}
\begin{verbatim}
subpath_within_superpath(subpath_results)
\end{verbatim}
\end{Usage}
%
\begin{Arguments}
\begin{ldescription}
\item[\code{subpath\_results}] Results data frame generated by
\code{\LinkA{subpathway\_analysis}{subpathway.Rul.analysis}}
\end{ldescription}
\end{Arguments}
%
\begin{Value}
A table with the proportion (and percent) of significant subpathways
within superpathways.
\end{Value}
%
\begin{Examples}
\begin{ExampleCode}

# Load data
data("demoDataSmall", package = "MetabolomicsPipeline")
dat <- demoDataSmall

# Runsubpathay analysis
sub_analysis <- subpathway_analysis(dat,
    treat_var = "GROUP_NAME",
    block_var = "TIME1",
    strat_var = NULL,
    Assay = "normalized"
)

#############################################################################
### Results Plots ###########################################################
##############################################################################

# significant subpathways by model type
subpath_by_model(sub_analysis)

# Percentage of signficant subpathways within superpathways
subpath_within_superpath(sub_analysis)

met_within_sub(sub_analysis, subpathway = "Aminosugar Metabolism")

\end{ExampleCode}
\end{Examples}
\printindex{}
\end{document}
